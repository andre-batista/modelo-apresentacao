% Esta é a classe para a produção de apresentações no LaTeX. Mais informações
% sobre o uso do Beamer podem ser encontradas em:
% * https://pt.overleaf.com/learn/latex/Beamer (Resumo em português)
% * https://texdoc.org/serve/beamer/0 (Documentação oficial)
\documentclass[11pt]{beamer}

% Configurações do tema: existem várias opções disponíveis, como Madrid, Berlin,
% Copenhagen, etc. Para mais informações, consulte a documentação do Beamer.
\usetheme{Madrid}
\usecolortheme{default} % default, beaver, seahorse
\usefonttheme{default} %  structurebold, structuresmallcapsserif, serif, default

% Cofigurações adicionais
\setbeamertemplate{caption}[numbered] % Numeração das figuras
\setbeamertemplate{frametitle continuation}{} % Separar as páginas de 
% referências sem contagem romana.

% Definição de pacotes
\usepackage[portuguese]{babel} % Palavras automáticas em português
\usepackage{graphicx} % Pacote para inserir figuras
\usepackage{subfig} % Pacote ideal para subfiguras
\usepackage[round,colon,sort]{natbib} % Pacote para referências bibliográficas
\usepackage{array} % Tamanho de colunas nas tabelas
\usepackage{xcolor} % Cores personalizadas
\usepackage{multirow} % Tabelas com multiplas linhas


% Exemplo de definição de cor personalizada
\definecolor{myblue}{HTML}{0071BC}

% Título: a parte entre colchetes é o título que aparece na barra
% superior ou inferior da apresentação, enquanto a parte entre chaves é o 
% título que aparece na primeira página.
\title[Trabalho de Conclusão de Curso]{Título do Trabalho de Conclusão de Curso}

% Subtítulo: seu trabalho pode ter um subtítulo, que é opcional. Se não for
% necessário, basta comentar a linha abaixo. Mas você pode colocar também
% a ocasião da apresentação, como "Defesa de Trabalho de Conclusão de Curso".
\subtitle{Defesa de Trabalho de Conclusão de Curso}

% Autor: a parte entre colchetes é o autor que aparece na barra superior ou
% inferior da apresentação, enquanto a parte entre chaves é o autor que aparece
% na primeira página. Os comandos \inst{}'s servem para indicar a filiação
% do autor, que é definida no comando \institute{}.
\author[Tal]{Fulano de Tal\inst{1}}

% Instituição: a parte entre colchetes é a instituição que aparece na barra
% superior ou inferior da apresentação, enquanto a parte entre chaves é a
% instituição que aparece na primeira página. Os comandos \inst{}'s servem
% para espcificar a filiação do autor usada anteriormente. Para TCCs, você
% pode usar o nome do colegiado do curso e da universidade. Para dissertações
% e teses, você pode usar o nome do programa de pós-graduação e da universidade.
\institute[Engenharia de Sistemas]{\inst{1}Colegiado do Curso de Graduação em 
Engenharia de Sistemas\\Universidade Federal de Minas Gerais}

% Data: a parte entre colchetes é a data que aparece na barra superior ou
% inferior da apresentação, enquanto a parte entre chaves é a data que aparece
% na primeira página. Você pode especificar manualmente a data da apresentação
% ou pode usar o comando \today que insere a data atual.
\date[24/07/2013]{24 de julho de 2013}

\begin{document}

    % Página de título
    \begin{frame}
        % O uso da logo da UFMG é totalmente opcional.
		\begin{flushright}
			\includegraphics[scale=.15]{./figuras/ufmg}
		\end{flushright}
		\maketitle
	\end{frame}

    % Sumário
    % Obs.: todo slide pode ter título e subtítulo. Não é obrigatório, mas pode
    % ser útil para organizar a apresentação. Pode seguir, por exemplo, a
    % definição das seções.
    \begin{frame}{Sumário}{Estrutura da apresentação}
		\tableofcontents[hideallsubsections]
	\end{frame}

    % Definição de Seções: toda definição de seção aparece automaticatimente no
    % sumário.
    \section{Introdução}

        % Slide de introdução
        \begin{frame}{Introdução}{Subtítulo da introdução}
            Você pode escrever normalmente dentro do ambiente do frame. O Beamer
            aceita todos os comandos do LaTeX, então você pode usar \textbf{negrito},
            \textit{itálico}, \underline{sublinhado}, etc. Além disso, você pode
            inserir figuras, tabelas, equações, etc.
        \end{frame}

        \begin{frame}{Introdução}{Como inserir tópicos}
            Lista de tópicos também podem ser utilizados. Os comandos são os mesmos
            do LaTeX, então você pode usar \texttt{itemize}, \texttt{enumerate} e
            \texttt{description} para criar listas.
            \begin{itemize}
                \item Item 1
                \item Item 2
                \begin{itemize}
                    \item Subitem 1
                \end{itemize}
                \item Item 3
            \end{itemize}
        \end{frame}

        \begin{frame}{Introdução}{Como inserir blocos de texto}
            \begin{block}{Bloco de texto}
                Este é um bloco de texto. Você pode usar blocos para destacar
                informações importantes, como definições, teoremas, etc.
            \end{block}
            \begin{alertblock}{Bloco de texto de outra cor}
                Este é um bloco de texto em outra cor.
            \end{alertblock}
            \begin{examples}{Bloco de texto de outra cor}
                Este é um bloco de texto de exemplo.
            \end{examples}
        \end{frame}

        \begin{frame}{Introdução}{Como inserir figuras}
            Figuras também podem ser inseridas. A definição do tamanho das imagens
            pode ser feita com o comando \texttt{scale}. No entanto, às vezes
            pode ser interessante definir o tamanho da figura em relação ao texto.
            Por exemplo, \texttt{width=.5\textbackslash textwidth} significa que a
            figura terá metade da largura do texto.
            \begin{figure}
                \centering
                \includegraphics[scale=.4]{./figuras/ufmg}
                \caption{Logo da UFMG}
            \end{figure}
        \end{frame}

        \begin{frame}{Introdução}{Como controlar espaçamentos}
            Às vezes pode ser necessário controlar espaçamentos entre os elementos
            da apresentação. Isso pode ser feito com o comando \texttt{vspace}.
            \vspace{1cm}

            Assim você pode aumentar a quebra de parágrafos ou a distância entre
            elementos. Também vale valores negativos, os quais diminuem o espaçamento.
        \end{frame}

        \begin{frame}{Introdução}{Como inserir tabelas}
            \begin{table}
                \centering
                \begin{tabular}{|c|c|c|}
                    \hline
                    \textbf{Coluna 1} & \textbf{Coluna 2} & \textbf{Coluna 3} \\
                    \hline
                    1 & 2 & 3 \\
                    4 & 5 & 6 \\
                    7 & 8 & 9 \\
                    \hline
                \end{tabular}
                \caption{Exemplo de tabela}
            \end{table}
        \end{frame}

        \begin{frame}{Introdução}{Como inserir equações}
            Todos os ambientes de equações do LaTeX podem ser utilizados no Beamer.
            Isso inclui \texttt{equation}, \texttt{align}, \texttt{gather}, \texttt{multline},
            \texttt{eqnarray}, etc.
            \begin{equation}
                \label{eq:exemplo}
                f(x) = x^2 + 2x + 1
            \end{equation}
            A equação \eqref{eq:exemplo} é um exemplo de equação quadrática.
            \begin{eqnarray}
                f(x) & = & x^2 + 2x + 1 \nonumber \\
                & = & (x + 1)^2
            \end{eqnarray}
        \end{frame}

        \begin{frame}{Introdução}{Como controlar o tamanho do texto}
            É bem possível que, às vezes, você precise diminuir ou aumenter o 
            tamanho do texto. Isso pode ser feito com os comandos \texttt{tiny},
            \texttt{scriptsize}, \texttt{footnotesize}, \texttt{small}, \texttt{normalsize},
            \texttt{large}, \texttt{Large}, \texttt{LARGE}, \texttt{huge} e \texttt{Huge}.
            Aqui vão alguns exemplos:
            \begin{itemize}
                \item \tiny{Texto em tamanho tiny}
                \item \scriptsize{Texto em tamanho scriptsize}
                \item \footnotesize{Texto em tamanho footnotesize}
                \item \small{Texto em tamanho small}
            \end{itemize}

        \end{frame}
    
    \section{Desenvolvimento}
    
            \begin{frame}{Desenvolvimento}{Subtítulo do desenvolvimento}
                O desenvolvimento é a parte principal do trabalho. Aqui você deve
                apresentar os resultados obtidos, as análises feitas, as discussões
                realizadas, etc.
            \end{frame}

            \begin{frame}{Desenvolvimento}{Como inserir subfiguras}
                \begin{figure}
                    \centering
                    \subfloat[Subfigura 1]{\includegraphics[width=0.4\textwidth]{./figuras/ufmg}}
                    \hspace{1cm}
                    \subfloat[Subfigura 2]{\includegraphics[width=0.4\textwidth]{./figuras/ufmg}}
                    \caption{Exemplo de subfiguras}
                \end{figure}
            \end{frame}

            \begin{frame}{Desenvolvimento}{Como inserir referências bibliográficas}
                \begin{itemize}
                    \item \citet{moulin2022induction} é um exemplo de referência.
                    \item \citep{torres2020neurocognitive} é outro exemplo de referência.
                \end{itemize}
            \end{frame}

            \begin{frame}{Desenvolvimento}{Como inserir colunas}
                \begin{columns}
                    \begin{column}{0.5\textwidth}
                        \begin{itemize}
                            \item Pode ser que você precise de duas colunas.
                            \item Nesse caso, você pode usar o ambiente \texttt{columns}.
                            \item O ambiente \texttt{column} define o tamanho de cada coluna.
                            \item Você pode usar \texttt{0.5\textbackslash textwidth} para metade da largura do texto. No caso de usar duas colunas.
                        \end{itemize}
                    \end{column}
                    \begin{column}{0.5\textwidth}
                        \begin{figure}
                            \centering
                            \includegraphics[width=0.8\textwidth]{./figuras/ufmg}
                            \caption{Logo da UFMG}
                        \end{figure}
                    \end{column}
                \end{columns}
            \end{frame}

    \section{Conclusão}

        \begin{frame}{Conclusão}{Subtítulo da conclusão}
            A conclusão é a parte final do trabalho. Aqui você deve apresentar
            as conclusões obtidas, as contribuições do trabalho, as limitações
            encontradas, as sugestões para trabalhos futuros, etc.
        \end{frame}

        \begin{frame}{Referências Bibliográficas}
            \bibliographystyle{plainnat}
            \bibliography{./referencias}
        \end{frame}

\end{document}